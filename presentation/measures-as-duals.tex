\documentclass{beamer}
\usepackage[english]{babel}
\usepackage[applemac]{inputenc}
\usepackage[lite,subscriptcorrection,slantedGreek,nofontinfo]{mtpro2}
\usepackage{mathtools}
\usepackage{amsthm}
\usepackage{amssymb}
\usepackage{amsmath}
\usepackage{extarrows}
\usepackage{enumitem}
 
\usetheme{Madrid}
\usecolortheme{beaver}
\usefonttheme{serif}%structuresmallcapsserif}

\AtBeginSection[]
{
  \begin{frame}
    \frametitle{Table of Contents}
    \tableofcontents[currentsection]
  \end{frame}
}

\AtBeginSubsection[]
{
  \begin{frame}
    \frametitle{Table of Contents}
    \tableofcontents[currentsubsection]
  \end{frame}
}

%Information to be included in the title page:
\title{\textsc{Weak compactness of probability measures}}%Reconstructing the marginal kernel of discrete determinantal point processes}}
\author{J. M�ller}
\institute{University of Warwick}
\date{\today}
 
 \setenumerate[1]{label=({\itshape\roman*})}
 \newcounter{ResumeEnumerate}
 \newtheorem{proposition}[theorem]{Proposition}
 %\newtheorem{corollary}[theorem]{Corollary}
%\renewcommand{\labelenumi}{\textup{(\textit{\roman{enumi}})}}
%\def\labelenumi{(\textit{\roman{enumi}})}
 
\begin{document}
%\def\labelenumi{(\textit{\roman{enumi}})}
 
\frame{\titlepage}

\begin{frame}
\frametitle{Table of Contents}
\tableofcontents
\end{frame}

\section{Motivation}

\section{Building a framework}

\begin{frame}
\frametitle{Setting and definitions}
Let \((X, \mathcal A)\) be a measurable space and let \(\mu\colon\mathcal A\to\mathbb R\). Then we say \(\mu\) is
\begin{enumerate}
\item \emph{finitely additive} if we have \[\mu\Big(\bigcup_{k=1}^N A_k\Big) = \sum\limits_{k=1}^N\mu(A_k)\] for all finite collection of measurable disjoint sets \(A_k\in\mathcal A\), \pause
\item \emph{countably additive} if the above holds for countable collections of measurable disjoint sets, \pause
\item of \emph{bounded variation} or shortly \emph{bounded} if we have
\[\left\lVert \mu \right\rVert_{BV}\coloneqq \sup_{\mathcal E}\sum\limits_{E\in\mathcal E} \left\lvert \mu(E) \right\rvert<\infty \]
where \(\mathcal E\) is any %going through %the supremum is taken over 
%all 
finite families of disjoint measurable sets, \pause
\item \emph{positive} if \(\mu(\mathcal A)\subseteq\mathbb R_+\). \pause
\end{enumerate}
Denote the space of bounded and finitely additive measures %with bounded variation 
by \(ba(\mathcal A)\) and the space of bounded and countably additive measures by \(ca(\mathcal A)\). %The quantity \(\left\lVert \mu \right\rVert_{BV}\) is called the norm of \emph{total variation} of \(\mu\) and Theorem \ref{charact} shows that it is indeed a norm and that both \(ba(\mathcal A)\) and \(ca(\mathcal A)\) are complete wrt to it.

\end{frame}

\begin{frame}
\frametitle{Hahn-Jordan decomposition}
\begin{theorem}[Hahn-Jordan decomposition]
For every signed measure \(\mu\in ba(\mathcal A)\) there are two positive measures \(\mu_+, \mu_-\in ba(\mathcal A)\) such that \(\mu = \mu_+-\mu_-\). If \(\mu\) is countably additive then  \(\mu_+\) and \(\mu_-\) can be chosen to be countably additive as well.
\end{theorem}\pause
\begin{proof}
\begin{enumerate}
\item For \(A\in\mathcal A\) set
\(\mu_+(A)\coloneqq \sup\left\{ \mu(B)\mid B\in\mathcal A, B\subseteq A\right\}.\) \pause
\item It is clear that \(\mu_+\) is positive as we have \(\mu_+(A)\ge \mu(\varnothing)=0\) and it can easily be checked that it is finitely additive. \pause
\item Choose now \(\mu_-\coloneqq \mu_+-\mu\) to obtain the desired decomposition. \pause
\item Use dominated convergence to show that \(\mu_+\) is countably additive if \(\mu\) is.
\end{enumerate}
\end{proof}
\end{frame}

\begin{frame}
\frametitle{Integration with respect to a signed measure}
\begin{theorem}[Integral]
For every signed measure \(\mu\in ba(\mathcal A)\) there is a unique linear and continuous mapping
\[\int\mathrm d\mu\colon L^0_b(\mathcal A)\to R\]
called the \emph{integral wrt to \(\mu\)} such that we have 
\[\int\limits \sum\limits_{k=1}^N \alpha_i\chi_{A_i} \mathrm{d}\mu \coloneqq \sum\limits_{k=1}^N \alpha_i \mu(A_i) \]
for all simple functions. Further we have
\[\left\lVert \int\mathrm d\mu \right\rVert_{L^0_b(\mathcal A)^\prime} \le \left\lVert \mu \right\rVert_{BV}.\]
\end{theorem}
\end{frame}

\section{Measures as dual spaces}

\begin{frame}
\frametitle{Characterisation of measures as a dual space}
\begin{theorem}[Measures as duals]
With the above notations the linear mapping
\begin{equation*}
I\colon ba(\mathcal A) \to L^0_b(\mathcal A)^\ast, \quad \mu\mapsto I\mu \coloneqq%F_\mu\coloneqq
 \left(f\mapsto\int f\mathrm{d}\mu\right)
\end{equation*}
is an algebraic isomorphism and we have \(\left\lVert I\mu \right\rVert_{L^0_b(\mathcal A)^\ast} = \left\lVert \mu \right\rVert_{BV}\). %In particular \(\left\lVert \cdot \right\rVert_{BV}\) is a norm and \(ba(\mathcal A)\) is complete with respect to it.
%isometric isomorphism.
 Further the inverse \(I^{-1}\) is given by
\[F\mapsto \Big( A\mapsto F(\chi_A)\Big).\]
For a measure \(\mu\in ba(\mathcal A)\) the following three statements are equivalent:
%is countably additive if and only if one of the two following equivalent properties holds:
\begin{enumerate}
\item \(\mu\) is countably additive.
\item \(I\mu\) fulfills the monotone convergence theorem.
\item \(I\mu\) fulfills the dominated convergence theorem.
\end{enumerate}
\end{theorem}
\end{frame}

\begin{frame}
\frametitle{Consequences}
\begin{corollary}
The quantity \(\left\lVert \cdot \right\rVert_{BV}\) is a norm and \(ba(\mathcal A)\) is complete with respect to it. The space of countably additive signed measures is a Banach space with respect to the norm of total variation.
\end{corollary}\pause
\begin{proof}
\begin{enumerate}
\item The first statement is an immediate consequence of the previous theorem. \pause
\item For the second one, show that strong limits of countably additive signed measures satisfy the dominated convergence theorem.
\end{enumerate}
\end{proof}
\end{frame}

\begin{frame}
\frametitle{Characterisation of \(L^\infty(\mu)^\ast\) and further duals}
\begin{theorem}[Dual of \(L^\infty(\mu)^\ast\)]
Let \(\mu\in ba(\mathcal A)\) and let \( ba(\mu)\subseteq ba(\mathcal A)\) be the subspace of measure that are absolutely continuous wrt \(\mu\). Then we have 
\[ ba(\mu) \cong L^\infty(\mu)^\ast \]
isometrically in the canonical way.
\end{theorem} \pause
\begin{proof}
Use the surjection
\[\iota:L^0_b(\mathcal A) \to L^\infty(\mu), \quad f\mapsto[f]_\mu\]
and the property \(\iota(B_1) = B_1\). 
\end{proof}
\end{frame}

\section{Weak topologies on the space of probability measures}

\begin{frame}
\frametitle{}

\end{frame}

\begin{frame}
\frametitle{}

\end{frame}

\end{document}

